\documentclass[a4paper]{report}
\usepackage[dutch]{babel} % Set language to Dutch.
\usepackage{graphicx}
\usepackage{enumerate}
\usepackage[utf8]{inputenc} % Set input to UTF-8 to enable easy input of umlauts etc.
\usepackage{hyperref} % Turn references into links.
\usepackage[hypcap]{caption} % Make figure references point to top of figure.

% Necessary to make \autoref command display nice Dutch words.
  \def\equationautorefname{Vergelijking}
  \def\footnoteautorefname{voetnoot}
  \def\itemautorefname{item}
  \def\figureautorefname{Figuur}
  \def\tableautorefname{Tabel}
  \def\partautorefname{Deel}
  \def\appendixautorefname{Appendix}
  \def\chapterautorefname{Hoofdstuk}
  \def\sectionautorefname{sectie}
  \def\subsectionautorefname{subsectie}
  \def\subsubsectionautorefname{subsubsectie}
  \def\paragraphautorefname{paragraaf}
  \def\subparagraphautorefname{subparagraaf}
  \def\FancyVerbLineautorefname{regel}
  \def\theoremautorefname{Theorema}
  \def\pageautorefname{pagina}

% Two new commands to display references a certain way.
\newcommand*{\fullref}[1]{\hyperref[{#1}]{\autoref*{#1}} \nameref*{#1}}
\newcommand*{\halfref}[1]{\hyperref[{#1}]{\ref*{#1}} \nameref*{#1}}

\makeindex
\begin{document}
\begin{titlepage}

\newcommand{\HRule}{\rule{\linewidth}{0.5mm}} % Defines a new command for the horizontal lines, change thickness here

\center % Center everything on the page
\vspace*{\fill}
 
%----------------------------------------------------------------------------------------
%	HEADING SECTIONS
%----------------------------------------------------------------------------------------

\textsc{\LARGE Universiteit Twente}\\[1.5cm] % Name of your university/college
\textsc{\Large Ontwerpproject}\\[2.0cm] % Major heading such as course name
%\textsc{\large Minor Heading}\\[0.5cm] % Minor heading such as course title

%----------------------------------------------------------------------------------------
%	TITLE SECTION
%----------------------------------------------------------------------------------------

\HRule \\[0.6cm]
{ \huge \bfseries Canvas.hs}\\[0.4cm] % Title of your document
{ \large \bfseries Event driven I/O voor Haskell met het HTML canvas}\\[0.4cm] % Title of your document
\HRule \\[1.8cm]
 
%----------------------------------------------------------------------------------------
%	AUTHOR SECTION
%----------------------------------------------------------------------------------------

\begin{minipage}[t]{0.4\textwidth}
\begin{flushleft} \large
\emph{Auteurs:}\\
J. \textsc{van Doorn}\\
L.J. \textsc{Buit}\\
P.T. \textsc{Jager}\\
M.J. \textsc{Scheepers}\\
M.J. \textsc{Roo}
\end{flushleft}
\end{minipage}
~
\begin{minipage}[t]{0.4\textwidth}
\begin{flushright} \large
\emph{Begeleider:} \\
E. \textsc{de Groote}
\end{flushright}
\end{minipage}\\[4cm]

% If you don't want a supervisor, uncomment the two lines below and remove the section above
%\Large \emph{Author:}\\
%John \textsc{Smith}\\[3cm] % Your name

%----------------------------------------------------------------------------------------
%	DATE SECTION
%----------------------------------------------------------------------------------------

\vspace*{\fill}
{\large \today} % Date, change the \today to a set date if you want to be precise

%----------------------------------------------------------------------------------------
%	LOGO SECTION
%----------------------------------------------------------------------------------------

%\includegraphics{Logo}\\[1cm] % Include a department/university logo - this will require the graphicx package
 
%----------------------------------------------------------------------------------------


\end{titlepage}
\tableofcontents

% !TEX spellcheck = nl_NL
\chapter{Inleiding}
Bij het informatica-keuzevak Functioneel Programmeren gebruiken studenten Haskell om fundamentele concepten van functioneel programmeren te bestuderen. Hierbij wordt door de studenten veel gebruik gemaakt van grafische weergaven om de werking van hun code inzichtelijk te maken. Het gebruik van Haskell voor het maken van grafische weergaven blijkt vaak redelijk gecompliceerd en limiteert studenten doordat zij zich bezig moeten houden met minder intuïtieve en minder essentiële aspecten van Haskell.

Om de focus binnen Functioneel Programmeren op de essentie te houden, is een grafische omgeving ontwikkeld op basis van de Gloss grafische bibliotheek. De interface tussen de code van de student en de grafische interface is eenvoudig en bruikbaar, het bevat alleen een aantal nadelen. Het werkt niet goed op ieder platform, mist een aantal functionaliteiten en de prestatie is niet uitstekend. In \autoref{hoofdstuk:requirements} wordt uitgebreid ingegaan op de requirements.

\subsubsection{Canvas.hs}
Het resultaat van dit ontwerpproject is \emph{Canvas.hs}; een omgeving die Haskell-gebruikers in staat stelt op eenvoudige wijze grafische elementen op een HTML5 canvas te presenteren. Canvas.hs is ontwikkeld met het oog op gebruiksgemak en eenvoud zonder de mogelijkheid tot uitbreiding en het toevoegen van geavanceerde functionaliteit onnodig te beperken.

\emph{Canvas.hs} is als library te installeren via \emph{Cabal}, de package manager voor Haskell programma's. In \fullref{sec:gebruikershandleiding} kan gelezen worden hoe de library gebruikt kan worden. Indien de libaray door een Haskell programma wordt aangeroepen zal een zeer lichte \emph{HTTP} en \emph{Websocket} server opgestart worden. Waarna de library een browser zal opstarten. Deze browser laad de \emph{JavaScript} code van de library in en toont vervolgens een \emph{HTML5 canvas} waar op getekend kan worden door het Haskell programma dat gebruik maakt van Canvas.hs. In \autoref{hoofdstuk:ontwerp} wordt het ontwerp van Canvas.hs beschreven.

De ontwikkelde library is uitgebreid getest. Dit geldt zowel voor de geschreven Haskell code als de JavaScript code. In \autoref{hoofdstuk:resultaten} kan hier meer over gelezen worden.

\subsubsection{Organisatie}
Bij de ontwikkeling van Canvas.hs is er gewerkt volgens eens vooraf vastgestelde afspraken. Daarnaast is er tooling toegepast om de samewerking tussen ontwikkelaars te stroomlijnen. In \autoref{hoofdstuk:organisatie} kan gelezen worden hoe dit alles is ingericht. Verder kan in \autoref{hoofdstuk:evaluatie} gelezen worden hoe de afspraken en tools uiteindelijk zijn toegepast.

\subsubsection{Uitbreiden}
Er zijn aanbevelingen voor de verdere verbetering van Canvas.hs. Hier wordt op in gegaan in \autoref{hoofdstuk:evaluatie} en \autoref{hoofdstuk:conclusie}. Verder is Canvas.hs opgezet zodat deze gemakkelijk is uit te breiden. In de \fullref{sec:uitbreiden} is te lezen hoe dit gedaan kan worden.






\chapter{Ontwerp} \label{hoofdstuk:ontwerp}

\section{Globaal ontwerp}  \label{sec:globaal}
Canvas.hs is een library die de programmeur kan importeren in zijn programma om er daarna met de \emph{API} die Canvas.hs aanbiedt gemakkelijk een uitgebreide user interface mee te bouwen. Canvas.hs focust zich op elementaire input en heeft geen ondersteuning voor high level interface elementen zoals knoppen en textgebeiden. Deze elementen zijn met behulp van Canvas.hs gemakklijk te implementeren.

\autoref{fig:overzicht_architectuur} geeft een overzicht van de architectuur. Canvas.hs bestaat uit een \emph{module} en een \emph{client}. De module is een library die de programmeur in zijn programma importeert. Bij het starten van de module start een \emph{HTTP} server, een WebSocket-server en wordt de webpagina van de client automatisch gestart. De client bestaat uit een browserpagina met onder andere een canvaselement. De module communiceert met de client via een \emph{WebSocket} verbinding om grafische elementen op de canvas in de client te tekenen.

\begin{figure}[H]
\begin{center}
\includegraphics[keepaspectratio,width=\textwidth]{./images/architecture_overzicht_poster.pdf}
\caption{Overzicht van de architectuur van Canvas.hs}
\label{fig:overzicht_architectuur}
\end{center}
\end{figure}

Een gebeurtenis die binnen het systeem plaatsvindt wordt een \emph{event} genoemd. Denk hierbij bijvoorbeeld aan een toetsaanslag of een muisklik. De user applicatie zal een eventHandler-functie moeten bevatten welke aangeroepen wordt na het plaatsvinden van een event. Deze eventHandler kan op basis van deze gebeurtenis een nieuwe grafische boom opleveren, hierbij kan gedacht worden aan een boom met daarin o.a. text en simpele vormen die samen een groter geheel vormen. Dit alles wordt hieronder toegelicht in \autoref{par:globaal_shapes}. De opgeleverde boom zal vervolgens op het canvas getekend worden. 

De eventHandler kan naast het opleveren van een nieuwe grafische boom ook nog een aantal uit te voeren acties opleveren; hierbij kan gedacht worden aan bijvoorbeeld het lezen of schrijven van bestanden. 

De eventHandler heeft de mogelijkheid om state bij te houden. Elke keer dat de eventHandler door Canvas.hs wordt aangeroepen krijgt deze de vorige opgeleverde state mee en de eventHandler geeft een nieuwe state in het resultaat. 

Het type van de eventHandler ligt bovenstaande duidelijk toe: \inlinecode{eventHandler :: a -> Event -> (a, Output)}, hierin is \inlinecode{a} de state die de eventHandler kan bijhouden. \inlinecode{Output} is een datatype waarin acties en een grafische boom gecombineerd kunnen worden.

\autoref{eventHandler_voorbeeld_simpel} illustreeert een simpele eventHandler die bij de start van het programma (\inlinecode{StartEvent}) een vierkant tekent en een timer start. Vervolgens wordt elke keer dat deze Timer afgaat het vierkant verplaatst. Als state wordt een simpele \inlinecode{Int} gebruikt. Door middel van de \inlinecode{installEventHandler} functie van Canvas.hs wordt de eventHandler geregistreerd en de grafische omgeving gestart.

\begin{lstlisting}[caption=Voorbeeld van een simpele eventHandler, label=eventHandler_voorbeeld_simpel]
import CanvasHs
import CanvasHs.Data

type State = Int

main = installEventHandler handle 0

handle :: State -> Event -> (State, Output)
handle st StartEvent    = (st+1, output)
	where 
		output = Out (Just shape, actions)
		shape = rectangle st
		actions = [Timer 1000 "move"]
		
handle st (Tick "move") = (st+1, shape $ rectangle st)
		
rectangle :: State -> Shape
rectangle i = Rect (10*i, 10*i) 10 10
\end{lstlisting}

Een gedetailleerde handleiding over Events, Shapes, Actions in het algemeen het gebruik van Canvas.hs kan gevonden worden in \fullref{sec:gebruikershandleiding}.

\paragraph{Shapes}
\label{par:globaal_shapes}
Zoals hierboven aangegeven levert de eventHandler o.a. een grafische boom op. Deze boom is gebasseerd op het Shapetype. Dit type definieert een aantal primitieven, zoals lijnen en vierkanten, en aantal mutaties zoals verschuivingen en rotaties die op een primitieve worden toegepast. Daarnaast wordt via het \inlinecode{Shape}-type ook aangegeven of er interesse is in events, zoals muisklikken, die plaats hebben op de Shape. In \autoref{dia:grafische_boom} is een grafische boom van Shapes weergegeven. Zoals te zien bestaan de bladen van de boom altijd uit primitieven en hebben de mutaties altijd één Shape als subboom.

\begin{diagram}
\Tree [.Container [.Fill [.Rotate [.{Event mouseClick=True} Rect ] ] ] [.Translate Text ] [.Stroke [.Container [.Circle ] [.Line ] [.Circle ] ] ] ]
\caption{Grafische boom}
\label{dia:grafische_boom}
\end{diagram}

\paragraph{Communicatie}
In Canvas.hs wordt de grafische weergave gedaan door het canvaselement in een webbrowser. Voor de communicatie tussen het haskellprocess en de webbrowser zijn een aantal mogelijkheden. In Canvas.hs hebben we ervoor gekozen in het Haskellprocess een server te starten die vervolgens door de webbrowser wordt aangesproken. Hier wordt over uitgeweid onder \autoref{subsec:architectuur}: architectuur.

\paragraph{Externe bibliotheken}
Voor het opzetten van deze server zijn vele oplossingen bedacht in de vorm van modules voor Haskell. Ook voor de interactie met het canvaselement bestaan een aantal JavaScriptbibliotheken.
In Canvas.hs hebben wij het wiel niet opnieuw uitgevonden, maar de beste van deze oplossingen geselecteerd. De gekozen bibliotheken worden verder toegelicht onder \autoref{subsec:externe_libraries}: externe bibliotheken.

\section{Detail ontwerp} \label{sec:detail}

In deze sectie wordt uitgebreid ingegaan op het ontwerp van de library. Eerst wordt de architectuur toegelicht in \autoref{subsec:architectuur} waar onder andere ingegaan wordt op de communicatie tussen de \emph{module} en de \emph{client}.

In \autoref{subsec:externe_libraries} wordt uiteengezet welke overwegingen hebben geleid tot het gebruik van verschillende extrerne libraries/dependencies. Daarna wordt dieper ingegaan op de werking van de \emph{module}, de \emph{client} en ondervonden problemen. Tot slot wordt de het ontwerp voor de werking van de grafische bibliotheek verder toegelicht in \autoref{subsec:grafische_bibliotheek}.

\subsection{Architectuur}
\label{subsec:architectuur}

De architectuur van een applicatie die Canvas.hs gebruikt bestaat uit drie componenten. De applicatie van de gebruiker, de Canvas.hs module en de JavaScript Canvas.hs applicatie. De Canvas.hs module biedt functies aan die de programmeur kan gebruiken om te tekenen en bepaalde acties uit te voeren in de browser. De module draait een server om te communiceren met de browseromgeving. De browseromgeving verbindt door middel van websockets met de server en geeft de output van het programma weer in een canvas HTML-element. In \autoref{fig:architectuur} is een schematische weergave van de architectuur weergegeven.

\begin{figure}
\begin{center}
\includegraphics[keepaspectratio,width=\textwidth]{./images/architecture.pdf}
\caption{Architectuur van Canvas.hs}
\label{fig:architectuur}
\end{center}
\end{figure}

\subsubsection{Communicatie}
Gegevensoverdracht tussen de HTTP server en de browser moet snel gebeuren. Om de grafische elementen in de browser weer te geven moeten de output van de grafische interface naar de browser gecommuniceerd worden. Wanneer de gebruiker interactie heeft met de interface moet dit naar het programma van de gebruiker gecommuniceerd worden. Verder zal het programma bepaalde acties moeten kunnen uitvoeren op de webbrowser, zoals fullscreen laten gaan van de browser. Een belangrijke overwegingen is dat de grafische interface zo min mogelijk vertraging moet hebben.


\paragraph{Websockets}
Het is mogelijk om door middel van XMLHttpRequest of WebSockets een verbinding te onderhouden tussen de Module en de Clientomgeving. XMLHttpRequests worden door alle webbrowsers ondersteund, en kan door middel van longpolling technieken (Comet) een verbinding onderhouden met de webserver. De meest recente browsers ondersteunen WebSockets. Dit biedt een socket verbinding tussen de client en de server. Het WebSockets protocol biedt een betere performance dan alle technieken op basis van XMLHttpRequests en biedt een groter implementatiegemak. Door het gebruik van het HTML canvas element zal de browser ondersteuning al beperkt zijn tot de meest recente browsers. Canvas.hs maakt gebruik van WebSockets.

\paragraph{Protocol}
Het formaat van het protocol is JSON\todo{JSON moet nog toegelicht worden}, dit is praktisch doordat dit eenvoudig in JavaScript te gebruiken is. In tegenstelling to XML heeft JSON bovendien weinig overhead.

Het protocol tussen de client en de server bevat voornamelijk interface data. De structuur en de attributen moeten vertaald worden van de Haskell omgeving om gebruikt te worden om te tekenen in het Canvas en vervolgens input van de gebruiker in de browseromgeving te ondersteunen. Voor optimale performance wordt de structuur zo snel mogelijk vertaald naar de structuur van KineticJS\todo{Kinetic moet ge\"introduceerd worden en toegelicht worden}. De programmeur bepaalt wanneer er interface data verstuurd word en wanneer acties verstuurd worden naar de client.

\paragraph{Stateless client}
In functioneel programmeren is het gebruikelijk\todo{gebruikelijk, veel gebruik te maken? dat is een beetje een vage aannamen} om veel gebruik te maken van functies die puur werken zonder state bij te houden in verschillende variabelen. In de huidige opzet van het vak functioneel programmeren biedt de grafische module een state object aan waarin de gebruiker alle state variabelen in kan opslaan. Wanneer de gebruiker de grafische interface updatet doet hij dit door de volledige boom van grafische elementen aan te passen of over de huidige output te tekenen. Dit maakt het overzichtelijk voor de programmeur omdat de grafische interface dan geen eigen state bijhoudt. Dit maakt de grafische interface op de client stateless. Een groot nadeel hiervan is dat iedere keer dat de grafische interface geupdated wordt ook de volledige grafische output weer naar de client verstuurd dient te worden. De enige state die de client vasthoudt wordt veroorzaakt door acties.

Een alternatief voor het versturen van volledige grafen is een vorm van delta updates, waarbij de programmeur opgeeft wat er veranderd is in de inteface of dat de module zelf uitvindt welke delen in de boom aangepast zijn. Beide mogelijkheden voegen veel extra complexiteit toe, voor respectievelijk de programmeur en voor het systeem. De bandbreedte van lokale websockets is groot en heeft geen last van latency waardoor de performance van de connectie hierdoor niet beperkt wordt.

\subsection{Externe libraries}
\label{subsec:externe_libraries}
\todo{blabla libraie sgebruikt mege leuk}

\subsubsection{HTTP-server}
Bij het starten van een applicatie met Canvas.hs wordt automatisch een browserscherm met daarop het canvas geopend. Deze pagina wordt samen met de benodigde javascript bestanden aan de browser geserveerd als statische content. Om deze content te kunnen serveren is een HTTP-server nodig. In Canvas.hs hebben we gekozen voor de ``warp'' module. De voornaamste redenen hiervoor zijn dat Warp lichtgewicht, goed gedocumenteerd en eenvoudig in gebruik is. Andere alternatieven zijn ``hapstack'', ``hyeana'' en ``snap server''. Deze bieden allen veel meer dan alleen de mogelijkheid om via HTTP bestanden te serveren en waren daardoor te veel en te ingewikkeld voor Canvas.hs

\subsubsection{Websockets}
In Canvas.hs wordt het resultaat van de user applicatie naar de javascript applicatie gestuurd via een webscoket. In Canvas.hs wordt gebruik gemaakt van de ``websockets'' module. Deze module is goed gedocumenteerd en wordt door de auteur goed onderhouden.

\subsubsection{JSON}
Voor het encoderen en decoderen van en naar JSON wordt in de Canvas.hs module de ``Aeson'' gebruikt. Hiermee is het relatief eenvoudig om datastructuren in Haskell om te encoderen naar JSON en vice-versa. Er is voor Aeson gekozen omdat deze goed gedocumenteerd is. Een alternatief voor Aeson was ``Text.JSON'', deze is niet gebruikt, omdat deze gebrekkig gedocumenteerd is.

\todo{Hier toevoegen over jQuery}
\todo{Hier toevoegen over Kinetic.js}

\subsection{Module}
De module is de Haskell bibliotheek die de programmeur gebruikt om de grafische interface in de client te bedienen. Naar de programmeur is de gebruiksvriendelijkheid van de bibliotheek een van de belangrijkste overwegingen voor het ontwerp. De module bestaat uit een aantal onderdelen: de server die de statische bestanden serveert, de websocket server die de verbinding met de client onderhoudt, en de laag die de input en output verwerkt. \autoref{fig:architecture_module} geeft de architectuur weer van de module.

\begin{figure}
\begin{center}
\includegraphics[keepaspectratio,width=\textwidth]{./images/module_architecture.pdf}
\caption{Architectuur van de module}
\label{fig:architecture_module}
\end{center}
\end{figure}

\paragraph{Servers}
Canvas.hs draait een simpele server op port 80 die statische bestanden kan serveren. Waaronder de index pagina, de javascript bestanden en eventueel plaatjes. Op port 8080 draait een websocket server die de verbinding met de client onderhoudt.


\paragraph{Server in de module}
De server draait in het proces dat gestart wordt vanuit de Haskell-code van de programmeur. De main van de programmeur start (indirect) de server. Dit is eenvoudiger dan het draaien van de server in een apart proces. Er hoeft namelijk niet tussen verschillende Haskell-processen gecommuniceerd te worden. Dit scheelt het schrijven van nog een interface tussen het server- en het module proces. Nadeel is wel dat het voortdurend opnieuw starten en afsluiten van de server leidt tot vertraging in het opstarten van het programma van de programmeur. Dit is vervelend als de programmeur regelmatig kleine wijzigingen maakt en dan de code opnieuw moet starten. Echter lijkt de overhead van het opnieuw starten van de server minimaal. Het is verder praktisch dat er geen rekening gehouden hoeft te worden met de state van de server bij het opstarten van het programma.

\paragraph{Gebruik} Wanneer de programmeur gebruik wil maken van de Canvas.hs moet hij gebruik maken van de installEventHandler functie. Bij het aanroepen van deze functie moet de programmeur een eventhandler meegeven die alle events vanuit de interface afhandelt. Om het gebruik van Canvas.hs zo makkelijk mogelijk te houden zal bij het aanroepen van installEventHandler automatisch de statische server en de websocket server gestart worden, en daarna automatisch de browserpagina geopend worden. \autoref{fig:startup_procedure} geeft de opstartprocedure weer.

\begin{figure}
\begin{center}
\includegraphics[keepaspectratio,width=\textwidth]{./images/module_startup_procedure_interaction.pdf}
\caption{De opstartprocedure en initiele interactiesequentie}
\label{fig:startup_procedure}
\end{center}
\end{figure}
\todo{Is het nou event handler of eventhandler?}
\paragraph{Input/output}
De module handelt input en output af door events naar de eventhandler van de programmeur te sturen. Bijvoorbeeld: wanneer een gebruiker op een rondje klikt zal het programma de eventhandler aanroepen met de ID van dat rondje en de locatie van de muisklik. De eventhandler van de programmeur kan dan nieuwe output genereren op basis van dit event. Zoals een nieuw menu weergeven of het uitvoeren van een actie zoals het opvragen van een bestand van de gebruiker.In \autoref{fig:startup_procedure} is deze interactie weergegeven.

De programmeur zal in zijn eventhandler bij ieder event de huidige state en de huidige event meekrijgen. Het type van de eventhandler is \inlinecode{userState -> Event -> (userState, Output)}, waarin de programmeur elk type aan userState kan geven. Door middel van pattern matching kan de programmeur makkelijk een bepaald event opvangen. Bijvoorbeeld: een muisklikevent wordt opgevangen door: \inlinecode{handler state (MouseClick (x,y) "id")}. De returnwaarde van de eventhandler is een tuple van de nieuwe state en de output. Output is een tuple van \inlinecode{(Maybe Shape, [Action])}. De eventhandler kan meerdere acties tegelijk uitvoeren en/of een grafische output leveren. De ondersteunde events en outputtypes worden verder toegelicht in \autoref{subsec:grafische_bibliotheek}.

\paragraph{Timers}
\todo{Als je ze in JS bijhoudt dan in Haskell verschillende threads? groot voordeel van wat?}
Canvas.Hs ondersteunt timers die ervoor zorgen dat met een bepaald interval een event naar de eventhandler wordt verstuurd. Hiermee kunnen bijvoorbeeld animaties worden toegevoegd aan de interface. Zoals het animeren van een bal of het animeren van een hele game. \todo{onduidelijk wat wordt hier mee bedoeld}Deze timers worden bijgehouden in de module. Een alternatief is om de timers aan de client over te laten. Dit heeft als voordeel dat het makkelijker is om timers bij te houden in JavaScript dan in Haskell. In Haskell moeten de timers per thread bijgehouden worden, in JavaScript gaat dit automatisch. Het voornaamste voordeel om het in de module te plaatsen is dat er geen extra acties over de WebSocketconnectie verzonden worden en dat een implementatie direct in Haskell meer precisie geeft voor de timer.

\paragraph{Unsafe I/O}
In Server.hs wordt gebruik gemaakt van unsafePreformIO voor het starten van child processen (een MVar die threads bijhoudt) en om een verbinding met een client bij te houden (een IORef die de connections naar de clients bevat). Hoewel het in dit geval volkomen veilig is, zijn er bezwaren tegen het gebruik van unsafePerformIO\cite{Haskell.org2008}. Dit is namelijk niet de netste oplossing en kan meestal voorkomen worden. Het is netter om een Server Monad te gebruiken die een State implementeert die deze zaken bijhoudt. Echter is het implementeren daarvan tijdsintensief, waardoor voor deze oplossing is gekozen.

\subsection{Client}
Het voornaamste onderdeel van de client is het canvas-element waarin de output getekend wordt door Kinetic.js. Daarboven zit de Canvas.hs javascriptcode die de verbinding onderhoud met de module en dit doorspeeld naar Kinetic.js om te tekenen. In dit deel worden voornamelijk de ontwerpbeslissingen besproken die betrekking hebben op de werking met KineticJS en de webbrowser.

\begin{figure}
\begin{center}
\includegraphics[keepaspectratio,width=\textwidth]{./images/client_architecture.pdf}
\caption{Architectuur van de JavaScript client}
\label{fig:architecture_client}
\end{center}
\end{figure}

\subsubsection{Canvas/Events}
Vanuit de module ontvangt de client via de WebSocket verbinding output. Ieder element wordt omgezet en aangemaakt in KineticJS. Voor elk element dat geinteresseerd is in een event wordt een eventlisteners toegevoegd. Wanneer de volledige structuur opgebouwd is wordt door oude structuur weggegooid en de nieuwe structuur op het canvas getekend.

\paragraph{Mousedrag}
Standaard ondersteund KineticJS mousedrag events, maar door de manier waarop Canvas.hs iedere keer opnieuw de output genereert raken de events van KineticJS verloren. Daarom maakt Canvas.hs gebruik van een eigen implementatie van mousedrag in de Canvas. Een mousedrag begint met een MouseDownEvent. Vanaf dat moment houd de client een ID bij van het huidige element. Alle opvolgende MouseMoveEvents zijn drag events die naar de client worden verstuurd. Deze MouseMoveEvents worden dooregegeven naar de Haskell kant met de vorige coördinaten van de muis en de huidige coördinaten van de muis.
Op deze manier kan de Haskell kant bepalen wat er moet gebeuren tijdens het draggen. Op het moment dat het programma aangeeft niet meer geinteresseerd te zijn in drag events op dat ID of in het geval van een MouseUpEvent stopt de drag in de client.

\subsubsection{Acties}
De programmeur kan een aantal specifieke acties uitvoeren op de client. De state van deze acties wordt onderhouden door de client zelf.
\todo{Referentie naar de specifieke sectie in de grafische bibliotheek}

\paragraph{Browserrestricties}
Canvas.hs bied de optie om de canvas in volledigscherm te laten weergeven. Door veiligheidsfunctionaliteiten in de huidige webbrowsers is het niet mogelijk om direct naar volledigscherm te gaan met behulp van JavaScript. Dit is alleen mogelijk vanuit klik- en toetsenbordevents. Daarom krijgt de gebruiker eerst een menu te zien voordat de browser naar volledigscherm gaat. Dezelfde veiligheidsrestricties zijn er voor het bestandenselectiemenu hierdoor heeft Canvas.hs daar ook een menu voor.

\todo{Referentie naar de grafische bibliotheek over volledigscherm}

\paragraph{Debug Console}
Voor de programmeur die gebruik gaat maken van de Canvas.hs library is het belangrijk dat zijn interface er zo uit ziet zoals hij dit wil. Ongetwijfeld zal een programmeur tegen problemen aanlopen bij het bouwen van de interface die hij niet had voorzien bij het schrijven van zijn code. Om probleemoplossing hiervan te vergemakkelijken bevat Canvas.hs een debug console bevat waar het aanroepen van teken functies en de invloed van deze API aanroepen goed visueel en tekstueel inzichtelijk worden. Doormiddel van een actie vanuit het programma van de gebruiker kan de debug console tevoorschijn gehaald worden.
\subsection{Haskell Interface} \label{subsec:grafische_bibliotheek} \todo{Titel kiezen die dezelfde lading dekt, maar minder engels is}
\todo{Deze subsection moet denk ik voor de 'problemen' subsection}

Onder \fullref{subsec:module} is toegelicht hoe de Canvas.hs-module is opgebouwd en hoe deze gebruikt kan worden. Een uitgebreide toelichting over het gebruik van Canvas.hs is te vinden in \fullref{sec:gebruikershandleiding}. Hieronder zullen enkele belangrijke ontwerpkeuzes voor de module worden toegelicht. Deze keuzes hebben grote invloed op de wijze waarop de module gebruikt moet worden. 

Hieronder zal o.a. worden toegelicht hoe tot de keuze voor de verschillende typen grafische elementen is gekomen, hoe gebeurtenissen aan deze elementen worden gekoppeld en hoe de acties die de gebruiker kan uitvoeren zijn opgebouwd.

\subsubsection{Grafische elementen}
De te tekenen grafische elementen zijn voor de gebruiker het belangrijkste onderdeel van Canvas.hs, er is daarom goed nagedacht over hoe de te tekenen grafische elementen worden gerepresenteerd. 

Grafische elementen worden in Canvas.hs gerepresenteerd door het \inlinecode{Shape}type. Dit type representeert alle mogelijk te tekenen grafische elementen en alle mutaties die daarop mogelijk zijn. Hier is al het e.e.a. over toegelicht onder \fullref{par:globaal_shapes}.

\paragraph{Mutaties}
Door deze representatie ontstaat een boom van shapes (\autoref{dia:grafische_boom}) waarin er shapes zijn die mutaties uitvoeren op andere shapes (zoals bijvoorbeeld \inlinecode{Rotate}) en primitieve shapes (zoals bijvoorbeeld \inlinecode{Rect} of \inlinecode{Circle}). Die mutaties die op shapes kunnen worden uitgevoerd worden dus zelf gerepresenteerd door een shape.

Een alternatief hiervoor is het uitvoeren van mutaties m.b.v. functies. Er zou dan bijvoorbeeld een functie \inlinecode{rotate :: Int -> Shape -> Shape} bestaan die een shape aanpast zodat deze geroteerd is. shapes zouden dan gerepresenteerd worden door een record-type waarin de functie aanpassingen maakt. 

De huidige oplossing had echter de voorkeur aangezien deze, in tegenstelling tot de oplossing m.b.v. functies, niet destructief is. Het is mogelijk om de mutaties weer uit de grafische boom te verwijderen en ze op deze manier ongedaan te maken. Daarnaast zou het gebruik van records voor de shapes voor problemen kunnen zorgen doordat de labels van velden in records in Haskell worden omgezet in functies. Hierdoor kan de gebruiker geen functies meer defini\"eren die de labels van record velden delen.

\paragraph{Positionering}
Voor het positioneren van grafische elementen op het canvas zijn vele mogelijkheden. In canvas.hs hebben we ervoor gekozen dat elk element relatief aan de bovenliggende \inlinecode{Container} wordt gepositioneerd. De bovenstaande \inlinecode{Container} wordt dan weer relatief aan het canvas gepositioneerd. 

Door deze manier van positioneren is het eenvoudig mogelijk om herbruikbare elementen, zoals knoppen e.d., te creeëren. Een functie levert een \inlinecode{Container} op die de knap bevat. Alle elementen van de knop (de text, de vorm van de knop zelf, etc.) zijn relatief aan die Container gepositioneerd. Door de Container vervolgens te positioneren worden alle elementen van de knop met deze container mee verplaatst. 

De keuze voor mutaties als shapes en relatieve positionering wordt hieronder geïllustreerd met een ``Mickey Mouse''-vorm die gedraaid en gekleurd is. De cirkels worden relatief aan de container gepositioneerd, vervolgens wordt de container zelf 100 pixels in de x en y richting verschoven m.b.v. de \inlinecode{Translate-Shape}. 

\begin{lstlisting}[style=densecode, language=Haskell]
Fill black $ Translate 100 100 $ Rotate 90 $ Container 200 200 [
            Circle (70, 130) 20,
            Circle (100, 100) 50,
            Circle (130, 130) 20
        ]
\end{lstlisting}


\subsubsection{Defaults}
Bij de \inlinecode{Text-} en \inlinecode{Event-Shape} wordt gebruik gemaakt van een record om opties mee te geven. In het geval van Text om bijvoorbeeld aan te geven of text dik- of schuingedruk moet worden en in het geval van Events om aan te geven in wat voor type Events interesse is. 

Een belangrijke overweging hierin was dat het voor de gebruiker mogelijk moet zijn om aan te geven dat er interesse is in bijvoorbeeld een mouseClick zonder expliciet aan te moeten geven dat er geen interesse is in bijvoorbeeld een mouseDrag. Met alleen records zou dit wel nodig zijn, aangezien daarbij van elk veld expliciet moet worden aangegeven welke waarde het heeft.

Dit probleem is opgelost met een \inlinecode{Defaults}-typeklasse. Deze definieert één enkele functie (\inlinecode{defaults :: a}) die een default waarde voor het gegeven type dient op te leveren. De gebruiker kan dan m.b.v. deze functie eenvoudig alleen de velden aanpassen waarin hij geïnteresseerd is. In het geval van Events zal de defaultsfunctie een record opleveren waarin voor elk type event stgaat de shape er niet in geïnteresserd is. 

In \autoref{mickey_event} wordt dat geïllustreerd. We breiden onze eerder gebruikte Mickey-vorm nu uit met een interesse in Events. met behulp van de defaultsfunctie krijgen we een record waarin voor elk event staat aangegeven dat er geen interesse in is. Vervolgens passen we dit record aan en geven we aan dat er interesse is in muiskliks.

\subsubsection{ID's}
Om op een praktische manier gebruik te kunnen maken van events worden deze aan shapes gekoppeld. Events die daadwerkelijk op een shape gebeuren (zoals muiklikken) kunnen dan ook identificeren op welke shape dat was. Hiertoe moeten shapes die geïnteresserd zijn in events geïdentificeerd kunnen worden. In Canvas.hs is besloten om de gebruiker, bij het aangeven van interesse in een event, een identifier mee te laten geven voor de shape. Hierdoor is het voor de gebruiker eenvoudig te zien op welke shape het event plaatsvond.

In \autoref{mickey_event} wordt dit geïllustreerd. Aan de eerder gedefiniëerde Mickey-vorm wordt interesse in een Event toegevoegd. De events die plaatsvinden op deze vorm zullen geïdentificeerd worden met de identifier "mickey".

\begin{lstlisting}[style=densecode, language=Haskell, caption=Mickey met interesse in een Event, label=mickey_event]
Event defaults{eventId="mickey", mouseClick=True} $ Fill black $ Translate 100 100 $ Rotate 90 $ Container 200 200 [
            Circle (70, 130) 20,
            Circle (100, 100) 50,
            Circle (130, 130) 20
        ]
\end{lstlisting}

Later in de handler zal de gebruiker bij events die op de Mickey plaatsvinden deze identifier meekrijgen en hier bijvoorbeeld op kunnen patternmatchen, zoals in onderstaande code geïllustreerd.

\begin{lstlisting}[style=densecode, language=Haskell]
handler :: State -> Event -> Output
(..)
handler state (MouseClick (x,y) "mickey") = -- Do something useful
\end{lstlisting}


\subsection{Protocol}
Uiteindelijk is ervoor gekozen om het protocol gelijkend aan Kinetic op te bouwen, zo zijn de meeste JSON berichten die gestuurd worden door de Canvas.Hs module een op een te teken op het canvas. Wel zijn er hier en daar kleine verschillen, JSON-keys die net anders zijn, of er zijn andere keys gebruikt om hetzelfde te modelleren (Een goed voorbeeld hiervan is de kleuren, kleuren worden opgestuurd als object met r, g , b en a waarden, maar Kinetic verwacht een string met een kleur zoals hij geaccepteerd wordt door een webbrowser. De manier hoe Kinetic omgaat met een kleur is een implementatiedetail dat niet thuis hoort in een protocol).
\todo{Niet zo veel tussen haakjes zetten}
\todo{dit aanpassen, ook stukje JSON toevoegen}

 
% !TEX spellcheck = nl_NL
\chapter{Conclusie \& Toekomstig werk} \label{hoofdstuk:conclusie}
Dit project had als doel om een nieuwe omgeving te ontwikkelen die beginnende Haskell-programmeurs, zoals studenten van het vak Functioneel Programmeren, in staat stelt om grafische weergaven te maken en te manipuleren met eenvoudige, zelfgeschreven Haskell-code. Deze omgeving zou bovendien op alle gangbare platformen moeten draaien terwijl de installatie van de omgeving niet onredelijk ingewikkeld moest zijn.

Canvas.hs, de ontwikkelde omgeving die in dit verslag gepresenteerd is, is de vervulling van dit doel. 

In \autoref{hoofdstuk:requirements} staan de requirements die aan de ontworpen omgeving gesteld werden en de traceability matrix in \autoref{sec:traceability} geeft aan welke requirements door welke tests geverifieerd worden.

\todo{Voltooi Conclusie}

Hieronder worden aspecten besproken die nog missen in Canvas.hs. Ook wordt per feature aangegeven hoe ondersteuning voor die feature toegevoegd kan worden in toekomstige versies.

\section{Verbeteringen en aanbevelingen}
\subsubsection{Image type}
Canvas.hs ondersteund een breed scala aan grafische primitieven, zo goed als alle vectortekeningen zijn te maken met deze library. Helaas mist er ondersteuning voor plaatjes, zo kan de gebruiker bijvoorbeeld geen fotoalbum in Haskell/Canvas.hs schrijven.

Het toevoegen van een \inlinecode{Image} type is daarentegen redelijk triviaal, al zal er nog wel over peformance nagedacht moeten worden. De eerste optie is dat de gebruiker het plaatje inleest (middels de \inlinecode{LoadFileBinary}) en vervolgens een \inlinecode{Image} node toevoegd die als argument een ByteString neemt, en dit aanbied aan Canvas.hs. Een andere optie is de foto te serveren vanuit de server, dus de gebruiker moet een image in een bepaalde map zetten, en de webserver serveert dat image dan op een statische manier.

Een groot nadeel aan de eerste methode is dat het plaatje steeds opnieuw naar base64 gedecodeerd moet worden, en over de socket gestuurd moet worden. Verder moet de gebruiker de bytestring van het plaatje bewaren, ander zou die steeds opnieuw ingeladen moeten worden. De tweede methode is flexibeler, Canvas.Hs stuurt alleen een locatie op naar de client, en de client kan zelfs cachen waar de foto opgeslagen is. Helaas betekend dit wel dat de gebruiker de foto waarschijnlijk in een specifiek mapje moet plaatsen.

Het implementeren van nieuwe shapes wordt in een van de appendices uitgebreid besproken.

\subsubsection{Kleurverlopen}
Binnen Canvas.hs kan tot op heden alleen met vaste keuren gevult worden, voor het tekenen van knoppen is het vaak gewenst iets van een kleurverloop te gebruiken. Een kleurverloop kan namelijk diepte bieden aan een grafisch element. Het ondersteunen van kleurverlopen is ook niet lastig, de clientside Javascript library (Kinetic) ondersteund al kleurverlopen, dus het is een kwestie van het toevoegen en implementeren van een API voor de gebruiker.

\subsubsection{Grafische toolkit}
\subsubsection{Delta updates}
Op dit moment wordt de client behandeld alsof het stateless is, de module stuurt na elke verandering de hele grafische boom op en de client gooit zijn hele interne boom weg en accepteerd de nieuwe. Deze manier van werken is vanuit een Haskell oogpunt logisch, maar door het gebrek aan caching is het in Javascript traag. Daartoe is het handig om delta updates te sturen, zo kan er opgestuurd worden dat een rondje drie keer zo groot geworden is. Helaas is dit nog niet triviaal, er moet bijvoorbeeld rekening gehouden worden met identieke nodes (bijvoorbeeld een cirkel op (0,0) met radius 10)
\subsubsection{FPprac module}

\chapter{Evaluatie} \label{hoofdstuk:evaluatie}
\todo{evaluatie tenopzichte van de requirements, of hoort dat in de conclusie?}

\section{Problemen}
\subsection{IO Monad afsplitsen}

\todo{Onderstaande had ik (Pim) origineel geschreven voor het problemenstuk (in ontwerp), het moet nog aangevuld/afgemaakt worden e.d., maar dat is iets wat bij evaluatie kan. Het is over de problemen die we hadden omdat we relatief weinig haskell-kennis hadden}

\subsubsection{Kennis Haskell}
Bij het begin van het Canvas.Hs-project was de kennis over Haskell en functioneel programmeren in het algemeen beperkt tot de kennis opgedaan met het vak Functioneel programmeren. Hoewel dit een solide basis vormt is het doel van Canvas.Hs juist om een aantal concepten die niet binnen dit vak passen af te schermen van de studenten. Dit betekende dat er bij het project van deze concepten gebruik moest worden gemaakt en wij ons deze ook eigen hebben moeten maken. Zoals altijd bij het leren van nieuwe concepten leverde dit af en toe code die niet optimaal gebruik maakte van de mogelijkheden van deze concepten en kinderziektes op. 

Naarmate het project vorderde vorderde ook onze kennis van Haskell, hierdoor maakt de uiteindelijke versie van Canvas.Hs goed gebruik van de mogelijkheden van o.a. monadisch programmeren en Haskells threadsysteem. 
\paragraph{Monads}
Zoals gezegd kent Haskell het concept van monads. Één van de doelen van CanvasHs is om dit concept niet te hoeven gebruiken voor grafische weergave bij het vak functioneel programmeren. Dit betekent dat ook wij, de ontwikkelaars, weinig kennis over dit concept hadden voor aan het project begonnen werd. Doordat we 
Denk aan do-notatie, binds (>>=
>>) en eigen monad voor de Server (state)
\todo{onvoldoende toegelicht}
\paragraph{Threads}
Denk aan geziek met netjes de WebSockets en http-threads afsluiten en Timer-threads die nu doorlopen

\todo{Hier is een stukje over het recordprobleem, maar waar het heen moet weet ik niet}
\subsubsection{Records}
Binnen Haskell is het zo dat elke sleutel van een record waarde een functie is. Als er een record gedefinieerd wordt met een sleutel ``sleutel'' bestaat er automatisch een functie \inlinecode{sleutel}. Hoewel dit op het eerste gezicht handig lijkt, betekend dit wel dat alle sleutels uniek moeten zijn. Het is zelfs zo dat sleutels globaal uniek moeten zijn, het is niet mogelijk twee records te maken (die niet in hetzelfde \inlinecode{data} element zitten) die dezelfde sleutel hebben.

De meestgebruike JSON libraries gebruiken records die daarna naar JSON vertaald worden, dit bleek problematisch. Omdat er meerdere records waren met dezelfde veldnamen moesten er handmatig toJSON instanties geschreven werden waar interne veldnamen omgezet werden in externe veldnamen. Uiteindelijk is er een oplossing gevonden in de vorm van Template Haskell, als de toJSON instances gemaakt worden door toJSON kan je aangeven dat veldnamen aangepast moeten worden, alle records zijn daartoe voorzien van een prefix die er door Template Haskell afgehaald wordt.



% !TEX spellcheck = nl_NL
\section{Planning}
\todo{Planning schrijven}
De planning tijdens dit project verliep volgens het scrum model zoals beschreven in \autoref{sec:scrum}. Dit model is flexibel en dynamisch waardoor 
\section{Organisatie}
De samenwerking is goed verlopen. Daarbij was de duidelijke structuur van zowel het project– als de technische organisatie een belangrijk onderdeel. Deze structuur is eerder beschreven in \fullref{hoofdstuk:organisatie}. Er kon met deze structuur, naar gevoel van de projectgroep, efficiënt gewerkt worden. Er werd wekelijks twee of meer keer samengekomen om te werken aan het project.

\paragraph{Besprekingen} Elke samenkomst is er een kortdurende scrumbespreking gehouden. Bij de dagelijkse besprekingen werd vaak op details in gegaan, hierbij werden ontwerp beslissingen vaak genomen tijdens deze besprekingen. De besprekingen duurde hierdoor soms wat lang en werden zittend gehouden. Met zogenaamde \emph{stand-up meetings} zou dit voorkomen konden worden. Bij deze besprekingen staat iedereen en zit niemand voor zijn/haar computer. Hiermee konden de dagelijkse besprekingen wellicht wat vlotter verlopen.

Na aanvang van het project zijn er ook een aantal sprint besprekingen gehouden. Deze sprint besprekingen kwamen altijd na het gesprek met de opdrachtgever. Er bleek al snel dat onze afgesproken sprint van twee werken wel erg kort was om elke twee weken dat er samengekomen werd een nieuwe lange sprintbespreking te houden. Dus er werd uiteindelijk voor gekozen om de besprekingen met de projectgroep te beperkingen tot de dagelijkse korte besprekingen.

\paragraph{Taken}
Regelmatig werd de lijst met taken bijgewerkt. Taken bestonden uit: projecttaken, als het bijhouden van de planning; ontwikkeltaken, als het oplossen van bugs en het schrijven van nieuwe features; schrijftaken en overige taken.

\paragraph{Burndownchart} Er is nu gewerkt in een projectgroep zonder direct uren budget, mocht dit het geval zijn zouden de inschattingsmethoden en burndown chart uit de scrum methode kunnen worden gebruikt.

\paragraph{Pull requests}
Bij aanvang van het project was deze werkwijze voor veel teamleden nieuw. En daarom is het in het begin niet altijd juist toegepast. Maar naarmate het project vorderde is steeds vaker met succes gebruik gemaakt van het pull request principe. Dit zorgde er voor dat teamleden elkaars code controleerde en dat er minder bugs zaten in de versies op de \inlinecode{dev} branch.

\paragraph{Verbeteringen} Naast het integreren van documentatie zou het buildscript ook automatisch per versie de test coverage uit kunnen rekenen. Dit zou ontwikkelaars nog meer inzicht kunnen geven in de kwaliteit van de software en hoe deze voor– of achteruit gaat per versie.






\newpage
\bibliographystyle{abbrv}
\bibliography{references}

\end{document}