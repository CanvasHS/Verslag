\subsubsection{Image type}
Canvas.hs ondersteund een breed scala aan grafische primitieven, zo goed als alle vectortekeningen zijn te maken met deze library. Helaas mist er ondersteuning voor plaatjes, zo kan de gebruiker bijvoorbeeld geen fotoalbum in Haskell/Canvas.hs schrijven.

Het toevoegen van een \inlinecode{Image} type is daarentegen redelijk triviaal, al zal er nog wel over peformance nagedacht moeten worden. De eerste optie is dat de gebruiker het plaatje inleest (middels de \inlinecode{LoadFileBinary}) en vervolgens een \inlinecode{Image} node toevoegd die als argument een ByteString neemt, en dit aanbied aan Canvas.hs. Een andere optie is de foto te serveren vanuit de server, dus de gebruiker moet een image in een bepaalde map zetten, en de webserver serveert dat image dan op een statische manier.

Een groot nadeel aan de eerste methode is dat het plaatje steeds opnieuw naar base64 gedecodeerd moet worden, en over de socket gestuurd moet worden. Verder moet de gebruiker de bytestring van het plaatje bewaren, ander zou die steeds opnieuw ingeladen moeten worden. De tweede methode is flexibeler, Canvas.Hs stuurt alleen een locatie op naar de client, en de client kan zelfs cachen waar de foto opgeslagen is. Helaas betekend dit wel dat de gebruiker de foto waarschijnlijk in een specifiek mapje moet plaatsen.

Het implementeren van nieuwe shapes wordt in een van de appendices uitgebreid besproken.

\subsubsection{Kleurverlopen}
Binnen Canvas.hs kan tot op heden alleen met vaste keuren gevult worden, voor het tekenen van knoppen is het vaak gewenst iets van een kleurverloop te gebruiken. Een kleurverloop kan namelijk diepte bieden aan een grafisch element. Het ondersteunen van kleurverlopen is ook niet lastig, de clientside Javascript library (Kinetic) ondersteund al kleurverlopen, dus het is een kwestie van het toevoegen en implementeren van een API voor de gebruiker.

\subsubsection{Grafische toolkit}
\subsubsection{Delta updates (met animaties)}
\subsubsection{Clientside animaties}
\subsubsection{FPprac module}
