\section{Samenwerking}
\todo{Dit moet naar het begin van het verslag}
De samenwerking is goed verlopen. Daarbij was de duidelijke structuur van zowel het project– als de technische organisatie een belangrijk onderdeel. Er kon met deze structuur, naar gevoel van de projectgroep, efficiënt gewerkt worden. Er werd wekelijks twee of meer keer samengekomen om te werken aan het project.

\subsection{Project organisatie}

\paragraph{Besprekingen} Elke samenkomst is er een kortdurende scrumbespreking gehouden. Bij de dagelijkse besprekingen werd vaak op details in gegaan, hierbij werden ontwerp beslissingen vaak genomen tijdens deze besprekingen. De besprekingen duurde hierdoor soms wat lang en werden zittend gehouden.


 Na aanvang van het project zijn er ook een aantal sprint besprekingen gehouden. Deze sprint besprekingen kwamen altijd na het gesprek met de opdrachtgever. Er bleek al snel dat onze afgesproken sprint van twee werken wel erg kort was om elke twee weken dat er samengekomen werd een nieuwe lange sprintbespreking te houden. Dus er werd uiteindelijk voor gekozen om de besprekingen met de projectgroep te beperkingen tot de dagelijkse korte besprekingen.

 \paragraph{Taken}
Regelmatig werd de lijst met taken bijgewerkt. Taken bestonden uit: projecttaken, als het bijhouden van de planning; ontwikkeltaken, als het oplossen van bugs en het schrijven van nieuwe features; schrijftaken en overige taken.

\paragraph{Verbeteringen} Er is nu gewerkt in een projectgroep zonder direct uren budget, mocht dit het geval zijn zouden de inschattingsmethoden en burndown chart uit de scrum methode kunnen worden gebruikt. Verder zouden de besprekingen zogenaamde \emph{stand-up meetings} kunnen worden waarbij iedereen staat en niet voor de computer zit. Hierdoor konden de dagelijkse besprekingen wellicht wat vlotter verlopen.

\paragraph{Pull requests}
Bij aanvang van het project was deze werkwijze voor veel teamleden nieuw. En daarom is het in het begin niet altijd juist toegepast. Maar naarmate het project vorderde is steeds vaker met succes gebruik gemaakt van het pull request principe. Dit zorgde er voor dat teamleden elkaars code controleerde en dat er minder bugs zaten in de versies op de \inlinecode{dev} branch.

\paragraph{Verbeteringen} Naast het integreren van documentatie zou het buildscript ook automatisch per versie de test coverage uit kunnen rekenen. Dit zou ontwikkelaars nog meer inzicht kunnen geven in de kwaliteit van de software en hoe deze voor– of achteruit gaat per versie.
