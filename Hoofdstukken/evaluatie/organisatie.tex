\section{Organisatie} \label{sec:organisatieevaluatie}
De samenwerking is goed verlopen. Daarbij was de duidelijke structuur van zowel het project- als de technische organisatie een belangrijk onderdeel. Deze structuur is eerder beschreven in \fullref{hoofdstuk:organisatie}. Er kon met deze structuur, naar gevoel van de projectgroep, efficiënt gewerkt worden. Er werd wekelijks twee of meer keer samengekomen om te werken aan het project.

\paragraph{Besprekingen} Elke samenkomst is er een kortdurende scrumbespreking gehouden. Bij de dagelijkse besprekingen werd vaak op details in gegaan, hierbij werden ontwerp beslissingen vaak genomen tijdens deze besprekingen.

Na aanvang van het project zijn er ook een aantal sprint besprekingen gehouden. Deze sprint besprekingen kwamen altijd na het gesprek met de opdrachtgever. Twee weken bleek een te kort interval om sprintbesprekingen te houden voor ons. Uiteindelijk is daardoor gekozen om de besprekingen met de projectgroep te beperken tot de dagelijkse korte besprekingen.

\paragraph{Pull requests}
Bij aanvang van het project werken met een branching model en pull requests voor veel teamleden nieuw. Daarom is deze aanpak in de eerste fases niet altijd juist toegepast. Naarmate het project vorderde is steeds vaker met succes gebruik gemaakt van het branchingmodel en pull requests. Dit zorgde er voor dat teamleden elkaars code controleerden en dat er minder bugs zaten in de versies op de \inlinecode{dev} branch.

\paragraph{Continuous Integration} Naast het compileren, testen en genereren van documentatie zou het buildscript ook automatisch per versie de test coverage uit kunnen rekenen. Dit zou ontwikkelaars nog meer inzicht kunnen geven in de kwaliteit van de software en hoe deze voor- of achteruit gaat per versie.
