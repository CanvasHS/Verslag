\chapter{Requirements} \label{hoofdstuk:requirements}


Het systeem bestaat uit een browseromgeving dat door middel van HTML en JavaScript een gebruikersinterface toont. Door middel van een verbinding tussen de browseromgeving en het Haskell-programma kan er vanuit het Haskell-programma output gegenereerd worden in de browser en kunnen er events verstuurd worden vanuit de browseromgeving naar het Haskell-programma.

Initieel is een lijst met requirements opgesteld. Gedurende het ontwikkelproces zijn bij de meetings met de begeleider nieuwe requirements aan de orde gekomen; deze requirements zijn in deze lijst opgenomen.

\subsubsection{Functionele requirements}
\begin{enumerate}[{R}1]
	\item Het systeem dient grafische primitieven zoals cirkels, vierkanten, lijnen, Bézier curves, n-hoeken en tekst moeten op een simpele manier getekend kunnen worden vanuit het Haskell-programma van de student.
	\begin{enumerate}[{R\arabic{enumi}.}1]
		\item Het systeem dient cirkels te kunnen tekenen.
		\item Het systeem dient vierkanten te kunnen tekenen.
		\item Het systeem dient lijnen te kunnen tekenen.
		\item Het systeem dient Bézier curves te kunnen tekenen.
		\item Het systeem dient tekst te kunnen tekenen.
		\begin{enumerate}[{R\arabic{enumi}.\arabic{enumii}.}1]
			\item Het systeem dient verschillende fonts te ondersteunen.
			\item Het systeem dient bold en italic tekst te ondersteunen.
			\item Het systeem dient verschillende font sizes te ondersteunen.
		\end{enumerate}
		\item Het systeem kan eventueel plaatjes inladen op het canvas.
	\end{enumerate}
	\item Het systeem dient vul- en lijnkleuren van grafische componenten instelbaar te maken.
	\begin{enumerate}[{R\arabic{enumi}.}1]
		\item Het systeem dient lijnkleuren instelbaar te maken.
		\item Het systeem dient vulkleuren instelbaar te maken.
		\item Het systeem kan eventueel gradients als vulkleur gebruiken.
	\end{enumerate}
	\item Het systeem dient events vanuit JavaScript door te geven aan het Haskell-programma van de student.
	\begin{enumerate}[{R\arabic{enumi}.}1]
		\item Het systeem dient toetsaanslagen vanuit de browser door te geven.
		\item Het systeem dient muisklikken vanuit de browser door te geven.
		\item Het systeem dient scroll-events vanuit de browser door te geven.
	\end{enumerate}
	\item Een programmeur moet grafische componenten aan kunnen passen zonder zijn programma te hoeven hercompileren.
	\begin{enumerate}[{R\arabic{enumi}.}1]
		\item Het systeem kan eventueel stapsgewijze aanpassingen geanimeerd weergeven.
		\item Het systeem kan eventueel zoomen en geschoven worden op de canvas.
	\end{enumerate}
	\item Het systeem kan acties ondersteunen die in de client of door de module worden uitgevoerd.
	\begin{enumerate}[{R\arabic{enumi}.}1]
		\item Het systeem kan eventueel tekstinvoer vragen met een pop-up.
		\item Het systeem kan eventueel fullscreen getoond worden.
		\item Het systeem dient lokale bestanden te kunnen verwerken.
		\begin{enumerate}[{R\arabic{enumi}.\arabic{enumii}.}1]
			\item Het systeem dient lokale bestanden te kunnen openen via een functie die aangeboden wordt in de Haskell-module.
			\item Het systeem dient lokale bestanden te kunnen wijzigen via een functie die aangeboden wordt in de Haskell-module.
		\end{enumerate}
		
		\item Het systeem kan eventueel bestanden die de gebruiker aanbied te kunnen verwerken.
	\end{enumerate}
	\item Het systeem dient duidelijke errors te genereren.
	\begin{enumerate}[{R\arabic{enumi}.}1]
		\item Als de verbinding tussen JavaScript en Haskell verbroken wordt dient het systeem een error te tonene aan de gebruiker. 
		\item Het systeem dient een error te tonen wanneer een functionaliteit in een onverwachte staat komt.
	\end{enumerate}
	\item Het systeem dient automatisch een browser te starten bij het draaien van het programma.
	\item Het systeem dient de browseromgeving opnieuw te laden als het Haskell-programma opnieuw gecompileerd wordt.
	\item De JavaScript-omgeving dient een simpele debug-console te bevatten.
	\newcounter{startvalue}
	\setcounter{startvalue}{\value{enumi}}
\end{enumerate}

\subsubsection{Niet functionele requirements}
\begin{enumerate}[{R}1]
\setcounter{enumi}{\value{startvalue}}
	\item Het systeem dient multiplatform te werken.
	\begin{enumerate}[{R\arabic{enumi}.}1]
		\item Het systeem dient te werken in de laatste versie van Internet Explorer, Firefox, Chrome en Safari.
		\item Het systeem dient te werken op een Linux, OS X en Windows.
		\item Het systeem dient te werken op een computer met een hoge pixeldichtheid.
	\end{enumerate}
	\item Het systeem dient gemakkelijk en snel te gebruiken te zijn.
	\begin{enumerate}[{R\arabic{enumi}.}1]
		\item Het systeem dient eenvoudig te installeren zijn.
		\item Het systeem dient makkelijk af te sluiten te zijn.
	\end{enumerate}
	\item Het systeem dient makkelijk te onderhouden zijn.
	\begin{enumerate}[{R\arabic{enumi}.}1]
		\item Het systeem dient gedocumenteert te zijn.
		\item Het systeem dient modulair opgebouwd te zijn.
	\end{enumerate}
	\item Het systeem dient getest te zijn met 70\% code coverage (exclusief frameworks).
\end{enumerate}