% !TEX spellcheck = nl_NL
\chapter{Inleiding}
Bij het informatica-keuzevak Functioneel Programmeren gebruiken studenten Haskell om fundamentele concepten van functioneel programmeren te bestuderen. Hierbij wordt door de studenten veel gebruik gemaakt van grafische weergaven om de werking van hun code inzichtelijk te maken. Het gebruik van Haskell voor het maken van grafische weergaven blijkt vaak redelijk gecompliceerd en limiteert studenten doordat zij zich bezig moeten houden met minder intuïtieve en minder essentiële aspecten van Haskell.

Om de focus binnen Functioneel Programmeren op de essentie te houden, is een grafische omgeving ontwikkeld op basis van de Gloss grafische bibliotheek. De interface tussen de code van de student en de grafische interface is eenvoudig en bruikbaar, het bevat alleen een aantal nadelen. Het werkt niet goed op ieder platform, mist een aantal functionaliteiten en de prestatie is niet uitstekend.

In dit ontwerpproject is \emph{Canvas.hs} ontwikkeld; een omgeving die Haskell-gebruikers in staat stelt op eenvoudige wijze grafische elementen op een HTML5 canvas te presenteren. Canvas.hs is ontwikkeld met het oog op gebruiksgemak en eenvoud zonder de mogelijkheid tot uitbreiding en het toevoegen van geavanceerde functionaliteit onnodig te beperken.

\subsubsection{Verslagstructuur}
In dit verslag wordt in \autoref{hoofdstuk:ontwerp} het ontwerp van Canvas.hs beschreven. Dit hoofdstuk is onderverdeeld in de secties: \halfref{sec:requirements}, \halfref{sec:globaal}, \halfref{sec:detail}, \halfref{sec:testplan} en \halfref{sec:testresultaten}. \autoref{hoofdstuk:conclusie} beschrijft de conclusies van dit project en in \autoref{hoofdstuk:evaluatie} wordt dit project met zijn uitkomsten geëvalueerd. Vervolgens wordt in \fullref{sec:gebruikershandleiding} een handleiding gepresenteerd die gebruikers van Canvas.hs aanwijzingen geeft over hoe de omgeving gebruikt en uitgebreid kan worden.


