\subsection{Architectuur}
De architectuur van een applicatie die Canvas.hs gebruikt bestaat uit drie componenten. De applicatie van de gebruiker, de Canvas.hs module en de JavaScript Canvas.hs applicatie. De Canvas.hs module bied functies aan die de programmeur kan gebruiken om te tekenen en bepaalde acties uit te voeren in de browser. De module draait een server om te communiceren met de browseromgeving. De browseromgeving verbind doormiddel van websockets met de server en geeft de output van het programma weer in een canvas HTML-element. In \autoref{fig:overzicht_architectuur} is een schematische weergave van de architectuur weergegeven.

\begin{figure}
\begin{center}
\includegraphics[keepaspectratio,width=\textwidth]{./images/architectuur_overzicht.pdf}
\caption{Overzicht van de architectuur van Canvas.hs}
\label{fig:overzicht_architectuur}
\end{center}
\end{figure}

\subsubsection{Communicatie}
Gegevensoverdracht tussen de HTTP server en de browser moet snel gebeuren. De gebruiker moet geen vertraging ervaren in de grafische interface.


\paragraph{Websockets}
Het is mogelijk om doormiddel van XMLHttpRequest of WebSockets een verbinding te onderhouden tussen de Module en de Clientomgeving. XMLHttpRequests worden door alle webbrowsers ondersteund, en kan doormiddel van longpolling technieken (Comet) een verbinding onderhouden met de webserver. De meest recente browsers ondersteunen WebSockets. Dit bied een socket verbinding tussen de client en de server. WebSockets bieden een betere performance dan alle technieken op basis van XMLHttpRequests en bied een groter implementatiegemak. Door het gebruik van het HTML canvas element zal de browser ondersteuning al beperkt zijn tot de meest recente browsers. Canvas.hs maakt gebruik van WebSockets.

\paragraph{Protocol}

Het formaat van het protocol is JSON, dit is praktisch doordat dit eenvoudig in JavaScript te gebruiken is. In tegenstelling to XML heeft JSON bovendien weinig overheid.

\subsubsection{Module}


\subsubsection{Canvas}



\subsubsection{Haskell}
Canvas.hs is een library die de programmeur kan importeren in zijn programma om er daarna met de API die Canvas.hs aanbiedt gemakkelijk een uitgebreide user interface mee te bouwen. Canvas.hs zal zich focussen op elementaire input en geen ondersteuning hebben voor high level interface elementen zoals buttons en textarea's. Deze elementen zouden met behulp van Canvas.hs wel eenvoudig te implementeren moeten zijn.

\paragraph{API}
De programmeur die gebruik maakt van Canvas.hs zal op een eenvoudige wijze gebruik kunnen maken van onze library. Daarom is het belangrijk dat er een API aangeboden wordt die eenvoudig te begrijpen is en niet teveel features bevat maar daarnaast wel de flexibiliteit biedt om een zeer complexe interface mee te bouwen.

\paragraph{Verbinding met de interface}
De verbinding met het canvas wordt bewerkstelligd met een eenvoudige HTTP server. Deze server biedt de gebruiker de mogelijkheid om via een webbrowser het Haskell programma te benaderen. De HTTP server biedt pagina's aan waarin JavaScript en HTML samenwerken om op het canvas te tekenen.

Als via de API van Canvas.hs begonnen wordt met tekenen zal de HTTP server automatisch gestart worden. Het besturingssysteem wordt aangeroepen voor het openen van de standaard browser --verwijzende naar het adres van de lokaal draaiende HTTP server.



\paragraph{Interface data}
De gegevens die door de HTTP server naar de browser gestuurd worden zullen voornamelijk bestaan uit interface data. Hoe, is de interface opgebouwd, elke elementen staan waar, welke attributen hebben deze elementen en zijn deze elementen in staat input van de gebruiker te accepteren. Hierbij zal er in het protocol rekening gehouden worden met de weergave van de elementen. De elementen die Kinetic.js ondersteund bieden hier een goede basis voor.

\paragraph{Events}
Events zijn de gegevens die de browser terug stuurt naar de HTTP server. Deze gegevens zullen vooral informatie bevatten over interface acties die de gebruiker uitvoerd. Het gaat hierbij om bijvoorbeeld muisbewegingen, muisklikken en toetsaanslagen. Het is belangrijk dat deze events snel door de server worden ontvangen en verwerkt naar nieuwe output zodat de gebruiker geen zichtbare vertraging ziet in de werking van het programma.

\subsubsection{Javascript}
Voor het tekenen van de interface op het HTML Canvas alsmede de communicatie vanuit de browser met de server, zal er gebruik worden gemaakt van JavaScript. jQuery biedt een prima uitbreiding op JavaScript waarbij de mogelijkheid wordt geboden verschillende zogenaamde jQuery plugins te gebruiken. In het geval van Canvas.hs zijn twee plugins belangrijk: Kinetic.js en jQuery-websockets.

Kinetic.js zal gebruikt worden voor het tekenen van de verschillende elementen op het canvas. Daarbij heeft Kinetic.js in compinatie met jQuery ook uitstekende ondersteuning voor het doorgeven van input events, die door bijvoorbeeld muisbeweginen aangeroepen worden. 

jQuery-websockets is een plugin die het gemakkelijk maakt gegevensoverdracht tussen websockets af te vangen en daar snel actie op te ondernemen. Deze lichtgewicht library maakt het mogelijk snel, zonder veel overhead opdrachten door te sturen naar Kinetic.js zodat deze op het canvas getekend worden.

\paragraph{Debug Console}
Voor de programmeur die gebruik gaat maken van de Canvas.hs library is het belangrijk dat zijn interface er zo uit ziet zoals hij dit wil. Ongetwijfeld zal een programmeur tegen problemen aanlopen bij het bouwen van de interface die hij niet had voorzien bij het schrijven van zijn code. Om probleemoplossing hiervan te vergemakkelijken bevat Canvas.hs een debug console bevat waar het aanroepen van teken functies en de invloed van deze API aanroepen goed visueel en tekstueel inzichtelijk worden.
